\documentclass[11pt]{article}
\renewcommand{\baselinestretch}{1.02}
\usepackage{times}
\usepackage{epsfig}
\usepackage{fullpage}
\usepackage{graphics}
\usepackage{txfonts}

\newcommand{\antijoin}{\rhd}

\setlength{\textheight}{9.5in}
\setlength{\topmargin}{-.2in}

\pagestyle{empty}

\newcommand{\answer}[2]{\noindent {\bf Answer:} #2}
\newcommand{\justanswer}[1]{\noindent {{\bf Answer:} #1}}
\newcommand{\mynewpage}{\newpage}
%\renewcommand{\justanswer}[1]{}
%\renewcommand{\answer}[2]{\vspace{#1}}
\renewcommand{\mynewpage}{}
\newcommand{\pt}[1]{\textbf{\underline{[#1 pts]}}}
\newcommand{\ptshort}[1]{\textbf{[#1]}}
\newcommand{\hard}{\textbf{[HARD]}}

\newcommand{\select}{{\bf select}\ }
\newcommand{\from}{{\bf from}\ }
\newcommand{\where}{{\bf where}\ }
\newcommand{\groupby}{{\bf group by}\ }
\newcommand{\orderby}{{\bf order by}\ }
\newcommand{\having}{{\bf having}\ }
\newcommand{\create}{{\bf create}\ }
\newcommand{\sqlexists}{{\bf exists}\ }
\newcommand{\sqland}{{\bf and}\ }

\newcommand{\spaces}{\mbox{\ \ \ \ \ \ \ \ }}

\newcommand{\miss}{{\em miss}\ }
\newcommand{\hit}{{\em hit}\ }

\newcommand{\commentout}[1]{}
\newcommand{\extracredit}[1]{{\bf [Extra Credit: #1 pts]}}

\newcommand{\fd}[2]{\ensuremath{#1 \rightarrow #2}}
\newcommand{\mvd}[2]{\ensuremath{#1 \rightarrow\rightarrow #2}}

\begin{document}
{
\begin{center}
%\large
%UNIVERSITY OF MARYLAND \\
%Department of Computer Science \\
%\vspace{5pt}
\large
\vspace{5pt}
\parbox{1.5in}{CMSC498O\\ Fall 2014} \parbox{3in}{\begin{center}\bf Introduction to Data Science I \\ MIDTERM \\ Closed Book\end{center}}
\parbox{1.5in}{Deshpande}
\end{center}
}

\begin{itemize}
\item Total points: 45. Weight: 15\% of the course grade. 
\item Show your reasoning. Write partial solutions. You will get a fair amount of the credit if I think you
know the concepts.

\item Unless otherwise specified, a {\em Yes/No} answer is {\em not sufficient} for any question. No points will be \\ given without
accompanying explanation.
\end{itemize}

\underline{\bf \large Your Name:} 



\newcommand{\qonetopic}{Misc Questions}
\newcommand{\qtwotopic}{Modelling}
\newcommand{\qthreetopic}{Functional Dependencies/Normalization}
\newcommand{\qfourtopic}{Relational Algebra/SQL}
\newcommand{\qfivetopic}{Buffer Management}

\newcommand{\er}{\textbf{(E/R Modeling)}}
\newcommand{\normalization}{\textbf{(Normalization)}}
\newcommand{\raid}{\textbf{(RAID)}}


%\section{\qonetopic \textbf{\ [3 pts each, 27 total]}}

\section*{Miscellaneous Questions 1 (9 questions - 2 pt each)}
\begin{enumerate}

\item What is ``volunteer bias'' in sampling? You can use an example.

\answer{1.9in}{See class notes.}

\item What are ``protocol buffers''? What are they used for?

\answer{1.9in}{See class notes.}

\item List three different data models.

\answer{1.7in}{See class notes.}

\item What is the difference between ``deduplication'' and ``record linkage'' in the context of entity resolution?

\answer{2.9in}{See class notes.}

\item Explain the rule-based approach to {\em relation extraction} in Information Extraction with an
example.

\answer{2.9in}{See class notes.}

\item Briefly explain the notion of ``overfitting'' in statistical modeling.

\answer{2.9in}{See class notes.}

\item List and briefly explain one classification technique.

\answer{2.5in}{See class notes.}

\item Briefly explain the ``local-as-view'' approach in data integration.

\answer{2.7in}{See class notes.}


\item Consider the relations: $R(A, B)$ and $S(B, C)$, and the SQL query: \\[3pt] {\tt select R.a, count(*) from R natural join S group by R.a;} \\[2pt]
Briefly explain the result of this query in words. Why might you want to use {\em left outer natural join} instead of {\em natural join} here ? 
Assume $A$ is a primary key for $R$.

\answer{2.9in}{The query returns for each tuple in $R$, the number of matching tuples in $S$. The result will not contain any tuples $t = (a, b)$ in $R$ that do not have a match in $S$, and we might want to use outerjoin to produce the result $(a, b, 0)$ for those tuples.}




\end{enumerate}

\section*{Miscellaneous Questions 2 (9 questions - 3 pt each)}
\begin{enumerate}

\item We want to test whether the average cost of the phones purchased by UMD students this year has gone up or down from previous year. Say we 
know that the average cost of a phone purchased last year by a UMD student was \$300. Explain the steps you would take to test 
the null hypothesis that the average cost this year = \$300, including how you would make the decision of rejecting or not rejecting the hypothesis.

\answer{3.9in}{
    \begin{itemize}
    \item Take a random sample of the UMD students who have a bought a phone in the current year (you can just take a random sample of the students,
            and ignore the students who didn't purchase a phone in the current year).
        \item Let $\mu$ denote the sample mean, and let $\sigma$ denote the sample deviation. Say $\mu > 300$.
        \item Choose a test statistic, in this case z-statistics is appropriate.
        \item Compute the z-statistic as: $\mu - 300/ \sigma$. 
        \item Consult standard formulas to decide how low is the probability of obtaining that z-statistic -- if it is too low (below 0.05), we
        reject the null hypothesis.
    \end{itemize}
}

\item Consider the following schema: \\
        create table r (a integer primary key, c integer); \\
        create table s (b integer primary key, a integer references r); \\
        create table t (c integer primary key, b integer references s); \\
        alter table r add constraint rreft foreign key (c) references t(c);

        \begin{itemize}
            \item Why can't I add the foreign key reference directly in the ``create table'' statement for table ``r'' ?

                \answer{.8in}{The table ``t'' hasn't been created yet.}
            \item Explain why the statement ``drop table r'' would be rejected.

                \answer{.8in}{Because there is a referential integrity constraint from ``s''.}
            \item Is there any way I can delete all the tables ? Explain in words.

                \answer{.8in}{Just reverse what we did to create the ``cyclic'' integrity constraints above. First alter the table ``r'' to 
                remove the referential integrity constraint, and then delete ``t'', and then ``s'', and finally ``r''.}
        \end{itemize}


\item What does `I' stand for in ACID properties? Briefly describe one mechanism for ensuring 'I'. 

\answer{2.7in}{See class notes}

\item List and briefly describe three single-source data quality problems.

\answer{2.7in}{See class notes}


\item Briefly, at a high level, explain the $k$-means clustering algorithm.

\answer{2.7in}{See class notes}



\mynewpage
\item Fill in the pseudocode for a naive implementation of the aggregation operation in the following query using Hashing.  \\[2pt]
{\tt select R.A, sum(R.B) from R group by R.A}

\begin{verbatim}
-----------------------------------------------------------------------------
HashMap h = new HashMap();
for each tuple r in R:
    if h.contains(r.A):
        h.replace(r.A, h.get(r.A) + r.B)
    else
        h.put(r.A, r.B)

// print out the results
for k in h.keySet():
    print k, h.get(k)

-----------------------------------------------------------------------------
\end{verbatim}

\item Consider an {\em email} dataset, containing the data about a collection of emails within a company. For each email, we know the email
addresses of the sender and the recipients, as well as the email text. There are however ambiguities since a person may
use different accounts, including external (e.g., gmail, or yahoo). Suggest three approaches you would use to disambiguate, i.e., decide which email
addresses belong to the same person. 

\answer{3.5in}{
Some options are:
    \begin{itemize}
        \item String similarity of the username part of the email address -- should only be used if the username is sufficiently distinct (e.g.,
                long usernames)
        \item Analyzing email texts can give fairly good indications about whether they were written by the same user.
        \item Users tend to send emails to the same group of users. We can generate some potential candidate pairs of email addresses (based on
        string similarity), and check if receivers of their emails overlap a lot (equivalently, we can check whether the people they receive
        emails from overlap.
    \end{itemize}
}



%\item The following two queries are not equivalent (they don't always produce identical results) because of NULLs. Identify and explain the problem. Schemas are: $R(a, b, d), S( c, d)$.  Assume $a$ is the primary key for $R$.
%
%\begin{table}[h]
%\begin{center}
%\begin{tabular}{|l|ll|}
%    \hline
%\underline{\bf Query I} \mbox{\ \ \ \ \ \ \ \ \ \ \ \ }\mbox{\ \ \ \ \ \ \ \ \ \ \ \ }&& \underline{\bf Query II}\mbox{\ \ \ \ \ \ \ \ \ \ \
%\ }\mbox{\ \ \ \ \ \ } \\
%{\bf select} a &&     {\bf select} a \\
%{\bf from} R &&   {\bf from} R, S  \\
%{\bf where} R.b = ({\bf select} {\bf count}(S.c) && {\bf where} R.d = S.d  \\
%\mbox{\ \ \ \ \ \ \ \ \ \ \ \ \ \ \ \ \ \ \ \ \ \ }{\bf from} S && {\bf group by} R.a \\
%\mbox{\ \ \ \ \ \ \ \ \ \ \ \ \ \ \ \ \ \ \ \ \ \ }{\bf where} R.d = S.d) && {\bf having} R.b = {\bf count}(S.c);\\
%    \hline
%\end{tabular}
%\end{center}
%\end{table}
%
%\answer{2in}{Consider a tuple in $R$: $(R.a = \alpha, R.b = 0, R.d = \beta)$, and say there is  no tuple in $S$ such that $S.d = \beta$. The first query will generate the answer tuple $\alpha$, whereas the second query will not. }
%
%
\item On the following tables, what are the results of the queries listed below?

\begin{table}[h]
\hspace{1in}{\bf R}\hspace{.2in}
    \begin{tabular}{|c|c|c|}
        \hline 
        {\bf A} & {\bf B} & {\bf C} \\
        \hline 
            'a1' & 10 & 10 \\
        \hline 
            'a1' & 20 & 20 \\
        \hline 
            'a2' & 30 & 30 \\
        \hline 
            'a2' & 0  & NULL \\
        \hline 
    \end{tabular}
\hspace{1in}{\bf S}\hspace{.2in}
    \begin{tabular}{|c|c|}
        \hline 
        {\bf C} & {\bf D} \\
        \hline 
        30 & 'd1' \\
        \hline 
        NULLL & 'd2' \\
        \hline 
    \end{tabular}
\end{table}

\begin{itemize}
\item {\tt select avg(B) from R group by A}:
\answer{.7in}{(15), (15)}
\item {\tt select * from R where C != 10}:
\answer{.7in}{Second and third rows would be returned, but not the fourth}
\item {\tt select * from R, S where R.C = S.C or R.C is null}: The result contains 3 tuples.
\answer{1in}{('a2', 30, 30, 30, 'd1'), ('a2', 0, NULL, 30, 'd1'), ('a2', 0, NULL, NULL, 'd2')}
\end{itemize}



\item (1) Consider an unbiased coin. Which of the two sequences, HHHHHH and HTTHTH, is more likely to happen if I were to roll the die 6 times?
Explain. \\
(2) An analyst for a fire department observed that: more firemen sent to control a fire meant higher overall damage. Should he
recommend that fewer firemen be sent to fires? Why or why not?
            

\answer{2.8in}{ 
(1) Both sequences are equally likely.

(2) No. Correlation does not equate causation. In this case, both the things (more firement sent to control the fire, and more damage) were likely caused by
the same underlying thing -- larger fires.
}

\end{enumerate}

\end{document}


